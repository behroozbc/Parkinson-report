\section{State-of-the-art}
The neurological state of PD patients is typically assessed from speech using regression analysis. This involves training a model to learn the relationship between acoustic features (extracted from speech signals) and the patient’s clinical score.\\ This is a list of researches on this subject.


\begin{itemize}
    \item Asgari and Shafran (2010) developed a method to predict the UPDRS-III score (motor sub-score) from speech recordings of 61 PD patients and 21 HC. They analyzed phonation, articulation, and prosody by extracting acoustic features from sustained vowel phonation, rapid syllable repetition, and reading standard texts. Features included F0, jitter, shimmer, spectral entropy, cepstral coefficients, and voiced/unvoiced frames. A SVR was trained to predict UPDRS scores, achieving a MAE of 5.66 using an ε-SVR with a cubic polynomial kernel.
    \item Tsanas (2010) conducted regression analysis to estimate UPDRS scores from 42 PD patients. Speech recordings of sustained vowel phonation were captured weekly for six months, while neurological assessments were done three times. Missing UPDRS scores were interpolated. Acoustic features based on pitch/amplitude perturbation, noise, and entropy were modeled. Regression analysis used least squares, LASSO, and CARTs. The CARTs approach performed best, with a MAE of 7.5 for total UPDRS and 6 for the UPDRS-III.
    \item Skodda (2013) evaluated speech deterioration over time in 80 PD patients through two recording sessions between 2002 and 2012. The time between the first and second sessions ranged from 12 to 88 months. A control group of 60 healthy individuals was also included. Participants read a text and produced sustained vowel phonation. Neurologists assessed patients using the UPDRS-III, and audio signals were analyzed for voice, articulation, prosody, and fluency. Significant differences in shimmer, speech rate, pause ratio, and Vowel Articulation Index (VAI) were found between the two sessions. However, results were inconclusive due to methodological limitations, such as the long interval between sessions.
    \item Bayestehtashk (2015) used ridge regression, LASSO regression, and linear SVR to predict UPDRS scores from speech recordings of 168 patients.
    \item Grosze (2015) The INTERSPEECH 2015 Computational Paralinguistic Challenge also addressed automatic acoustic analysis of PD. The challenge involved predicting MDS-UPDRS-III scores using recordings of 50 patients for training and development, and 11 new patients for testing in non-controlled noise conditions. The winners achieved a Spearman’s correlation of 0.65 between actual and predicted MDS-UPDRS-III scores using deep rectifier neural networks and Gaussian processes.
    \item Orozco-Arroyave (2016) developed a methodology to estimate the neurological state (MDS-UPDRS-III) of 158 PD patients from Colombia, Germany, and the Czech Republic using linear ε-SVR. Speech tasks included reading isolated words, sentences, a standard text, and a monologue. Articulation problems were modeled by analyzing energy transitions between voiced and unvoiced segments. Speech intelligibility was assessed using an automatic speech recognition system. The methodology achieved a Spearman’s correlation of up to 0.74 for estimating the neurological state when combining articulation and intelligibility measures.
    \item Eyben (2010) The openSMILE toolkit was used for feature extraction, computing over 6000 descriptors. The authors reported that the neurological state of patients could be assessed with a MAE of 5.5.
    \item Gomez-Vilda (2017) conducted a study to monitor PD progression using speech recordings from 8 male patients, captured twice over four weeks, and 100 healthy speakers as a baseline. Participants performed sustained vowel phonation and read a short sentence and standard text. Features were estimated using vocal tract inversion and biomechanical inversion of a 2-mass vocal fold model. Features included jitter, shimmer, harmonicity, vocal fold mechanical stress, and tremor. Patients continued pharmacological treatment and received speech therapy. Evaluations were based on the H\&Y scale, and the relationship between neurological scale and acoustic features was assessed using Bayesian Likelihood. The study found that tremor and biomechanical features evolved differently with treatment, suggesting the need for different time intervals between evaluations for more conclusive results.
    \item Sztaho (2017) developed a method to estimate PD severity using rhythm-based features from speech recordings of 51 PD patients and 27 healthy speakers. Patients were evaluated using the H\&Y scale. Speech tasks included a monologue and reading a standard text. Rhythm features analyzed included consonant/vowel duration, speech/pause duration, pause ratio, articulation rate, and Pairwise Variability Index (PVI). Regression analysis using linear regression, SVR, ANN, and DNN achieved a Spearman’s correlation coefficient of up to 0.744 (SVR, reading task) between predicted and target H\&Y scores.
    \item Hemmerling and Wojcik-Pedziwiatr (2020) estimated PD severity by extracting acoustic features from sustained vowel phonation (/a/, /e/, /i/, /o/, /u/). Features included average F0, jitter, shimmer, energy, spectral moments, MFCCs, and PLP coefficients. Speech recordings of 27 PD patients from Poland were captured five times over 180 minutes after levodopa medication. A neurologist estimated UPDRS scores during the sessions. Motor UPDRS scores were estimated using multiple linear regression, Random Forest (RF) regression, and SVR. The lowest error (MAE=1.85) was obtained for the vowel /a/ using RF regression.
    \item Cernak (2017) evaluated voice quality changes in speakers using the mFDA score for larynx deficits. They trained an SVR with phoneme posterior probabilities from recordings of 50 PD patients and 50 HC from Colombia. Speech tasks included rapid syllable repetition, reading a standard text, and a monologue. The study reported Spearman’s correlation coefficients of up to 0.57 between predicted scores and the larynx mFDA score.
    \item García (2017) predicted the neurological state (MDS-UPDRS-III) and mFDA of 50 PD patients using acoustic analysis of pitch, loudness, duration, and filterbank parameters from four speech tasks. The i-vector approach was used to obtain speaker models of 50 PD patients and 50 HC from Colombia. The study achieved a Spearman correlation of 0.63 for MDS-UPDRS-III prediction using phonation and articulation features from sentences, and 0.72 for mFDA prediction using phonation, articulation, and prosody features from rapid syllable repetition.
    \item Vasquez-Correa (2018) conducted a study to estimate the dysarthria level in 68 PD patients and 50 HC from Colombia. They used various speech tasks, including sustained phonation, reading sentences, a monologue, and rapid repetition of specific syllables. Automatic acoustic analysis was performed using i-vector speaker models based on phonation, articulation, prosody, and intelligibility features. They applied different regression techniques to estimate mFDA scores. The highest Spearman’s correlation coefficient was 0.69 for articulation features from continuous speech.
    \item Karan (2020) conducted a study to estimate the mFDA score of 70 PD patients. They used speech recordings of sustained phonation of vowels (/a/, /e/, /i/, /o/, /u/) and the reading of 10 isolated words. They combined F0 and Hilbert’s spectral features and performed regression analysis using ε-SVR. The highest Spearman’s correlations reported were 0.75 for the vowel /o/ and 0.77 for the word “reina” (which means “Queen”).
\end{itemize}

In general, the sustained phonation of vowels and reading a standard text are commonly used speech tasks to assess a patient’s neurological state. These tasks help detect speech problems, with reading tasks specifically evaluating articulation and prosody issues. Common biomarkers for modeling speech problems include pitch (F0, jitter), harmonicity (e.g., harmonics-to-noise ratio), and spectral energy. Support Vector Regression has been effective in modeling the relationship between acoustic features and clinical scores.
