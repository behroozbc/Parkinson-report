\section{Speech production process}
\subsection{ Speech chain}
The speech chain model by Denes and Pinson (1993) describes oral communication as a sequence of events on three levels: linguistic, physiological, and acoustic. It starts in the speaker’s brain (linguistic level) where thoughts are formed into sentences.
Neural activity then activates the muscles controlling the vocal folds, tongue, lips, and jaw (physiological level).
Finally, speech sound waves are generated by these muscle movements and air from the lungs (acoustic level).
The produced speech travels through the air to the listener’s hearing mechanism.
Auditory feedback helps speakers monitor the quality and intelligibility of their speech.\\
\textbf{Physiological processes of speech production}\\
The speech production process involves the coordination of muscles in the respiratory, laryngeal, and oral motor systems. The respiratory system generates air pressure from the lungs, which passes through the glottis formed by the vocal folds. During phonation, the vocal folds vibrate, creating voiced sounds. If the vocal folds remain open, unvoiced sounds are produced by other articulators. The oral motor system, including the tongue, jaw, lips, and velum, modulates the voice source to produce different speech sounds. Nasal, oral, and pharyngeal cavities act as resonance chambers. For example, the vowel /a/ is produced by the combined movements of the tongue, jaw, and vocal folds, while plosive sounds like /p/ are produced by blocking and releasing the air stream with the lips. Nasal sounds like /n/ and /m/ involve the vocal folds and blocking the air stream in the oral cavity.

\subsection{ Impact of PD on speech motor control}

Motor deficits in PD involve the interaction of the basal ganglia, motor cortex, and thalamus. The basal ganglia include structures such as the striatum, Globus Pallidus, subthalamic nucleus (STN), and substantia nigra. Motor impairments in PD are mainly caused by the degeneration of dopaminergic neurons in the substantia nigra. The basal ganglia send signals to the thalamus, influencing motor cortex activity. The basic circuit model of the basal ganglia, proposed by Albin et al. (1989), remains valid for understanding motor control in PD, despite more complex connections being discovered since then. The baseline of the circuit model involves two main parallel loops:
\begin{enumerate}
    \item The first loop is a cortex-to-cortex circuit where the motor cortex sends signals to the striatum. These signals then travel to the globus pallidus, continue to the thalamus, and finally return to the motor cortex.
    \item The second loop involves the substantia nigra projecting dopaminergic neurons to the striatum, causing two opposite effects on D1 and D2 dopamine receptors: excitation (in D1) and inhibition (in D2).
\end{enumerate}
The regulation of movement excitation and inhibition involves dopaminergic input to the striatum, affecting the basal ganglia via \textbf{direct and indirect pathways}. The direct pathway excites the motor cortex to facilitate movement, while the indirect pathway inhibits motor activity to suppress involuntary movement. Dopamine regulates neuron excitability in the striatum, balancing the direct and indirect pathways. In PD patients, decreased dopamine levels disrupt this balance, leading to increased inhibition in the indirect pathway and decreased inhibition in the direct pathway, resulting in difficulty controlling movements.\\
\textbf{Motor speech disorders in PD}\\
Hypokinetic dysarthria, a speech production disorder associated with PD, results from dysfunction in the basal ganglia pathways. It is characterized by reduced movement range, rigidity, and slow repetitive movements, affecting phonation, articulation, and prosody. Phonation issues include breathiness, hoarse speech, voice tremor, and vocal cord bowing. Articulation problems involve prolonged speech sounds, difficulty initiating speech, and imprecise articulation. Prosodic issues include reduced pitch and loudness variability, rapid speech rate, and reduced loudness. These deviations are detected during conversational and read speech tasks.
